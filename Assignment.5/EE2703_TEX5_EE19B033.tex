\documentclass[11pt, a4paper, twoside]{article}
\usepackage{graphicx}
\usepackage{amsmath}
\usepackage[margin=0.8in]{geometry}
\usepackage{listings}
\usepackage{float}
\usepackage{fancyhdr}
\usepackage{indentfirst}
\usepackage[inline]{enumitem}
\usepackage{xcolor}
\usepackage{minted}
\usemintedstyle{borland}
\usepackage[belowskip=0pt,aboveskip=0pt,font=small,labelfont=small]{caption}
\captionsetup{width=0.9\linewidth}
\setlength\intextsep{0pt}

\setlist[itemize]{noitemsep, topsep=0pt}
%\fancyhead[RO,LE]{EE2703: Assignment 5}

%\cfoot{\thepage}

\title{EE2703: Assignment 5}
\author{KORRA SRIKANTH\\ {\small EE19B033}}
\date{\today}

%\pagestyle{fancy}
\begin{document}	
	
\maketitle % Insert the title, author and date		
\section*{Introduction}
 We wish to solve for the currents in a resistor.
 The currents depend on the shape of the resistor.
 we also want to know which part of the resistor is likely to get hottest.

 \section*{Aim:}
  \textbf{The goal of this assignment is the following:}
  \begin{itemize}
  \item
  	To solve for currents in a system.
  \item
  	To solve 2-D Laplace equations in an iterative manner.
  \item
	To understand how to vectorize code in python.
  \item
	To plot graphs to understand the 2-D Laplace equation.	
\end{itemize}  
  
\section*{Task 1:\\ Defining parameters and Initializing the potential matrix:}

\begin{itemize}
\item Asking the user to pass parameter values  \(N_x\) , \(N_y\) and r.
\item
  Define the Parameters, The parameter values taken for my particular code were \(N_x\) and \(N_y\) and No of iterations : 1500
\item
  These values are taken to discuss about Stopping condition,etc
\item
  To allocate the potential array \(\phi = 0\) .Note that the array
  should have \(N_y\) rows and \(N_x\) columns.
\item
  To find the indices which lie inside the circle of radius(r) using
  meshgrid() by equation :
\end{itemize}

\begin{equation}
X ^2 +Y ^2 \leq	 r^2
\end{equation}

\begin{itemize}
\item
  Then assign 1v to those indices.
\item
  To plot a contour plot of potential \(\phi\) and to mark V=1 region in
  red
\end{itemize}

\subsection*{Code:}
\begin{verbatim}
if (len(argv) ==4):
    Nx = int(argv[1])         # No. of steps along the x direction
    Ny = int(argv[2])         # No. of steps along the y direction
    r = float(argv[3])
    print("Using user provided params")
else:
    Nx = 25
    Ny = 25  
    r = 8
    print("Using default parameters") 

phi = zeros((Nx,Ny))
Niter = 1500
x = linspace(-Nx/2,Nx/2,Nx)
y = linspace(Nx/2,-Nx/2,Ny)
Y,X = meshgrid(y,x)

# Finding out the points with 1v potential
Z = where(X*X + Y*Y <= r*r )

# Marking potential = 1V points in the phi matrix 
for i in range(Z[0].size):
    phi[Z[0][i],Z[1][i]] = 1
  
# Plotting the 1V points and the contour plot of phi matrix
figure(1) 
title('1V Potential and the contour plot of potential matrix')
xlabel(r'$x$')
ylabel(r'$y$')
plot(x[Z[0]],y[Z[1]],'ro', label = '1V Potential')
contour(X,Y,phi)
legend()
grid()
show()   
\end{verbatim}

	     \begin{figure}[!tbh]
        \centering
        \includegraphics[scale=0.8]{Figure_51.png}  
        \caption{Contour plot of initial potential}
   \end{figure}
   
  
  
\section*{Task 2 : \\Performing the Iterations:}

\begin{itemize}
   \item To update the potential \(\phi\) according to Equation below using
     vectorized code
   \end{itemize}
   
   \begin{equation}
           \phi_{i,j} = \frac{\phi_{i+1,j} + \phi_{i-1,j} + \phi_{i,j+1} + \phi_{i,j-1}}{4} 
   \end{equation}
   
  
   \begin{itemize}
   \item
     To plot the errors in semilog and loglog and observe how the errors
     are evolving.
   \end{itemize}

\subsection*{Code:}
\begin{verbatim}
# iteration which calculates the steady state potential array
error=[]
e1 = zeros((30))
# iteration which calculates the steady state potential array
for  i in range(Niter) :
    oldphi = phi.copy()
    phi[1:-1,1:-1] = 0.25*(phi[1:-1,0:-2]+phi[1:-1,2:]+phi[0:-2,1:-1]+phi[2:,1:-1])
    phi[1:-1,0]=phi[1:-1,1]
    phi[1:-1,Nx-1]=phi[1:-1,Nx-2]
    phi[0,1:-1]=phi[1,1:-1]
    phi[0,0]=phi[0,1]
    phi[0,Nx-1]=phi[0,Nx-2]
    phi[Ny-1,1:-1]=0.0
    phi[Z]=1.0
    error.append((abs(oldphi-phi)).max())
\end{verbatim}

     \begin{figure}[!tbh]
      \centering
      \includegraphics[scale=0.8]{Figure_52.png}\\
      %\includegraphics[scale=0.8]{Figure_53.png}  
      \caption{Semilog plots of Error vs No.of Iterations}
 \end{figure}
 
  \begin{figure}[!tbh]
      \centering
     % \includegraphics[scale=0.8]{Figure_52.png}\\
      \includegraphics[scale=0.8]{Figure_53.png}  
      \caption{Log-Log plots of Error vs No.of Iterations}
 \end{figure}
 
 
 \newpage
 
\section*{Task 3 :\\ Fitting the exponential function to the error plots:}

\begin{itemize}
\item
  To find the fit using Least squares for all iterations
  and for iterations \(\geq\) 500
  separately and compare them.
\item
  As we know that error follows \(Ae^{Bx}\) at large iterations, we use
  equation given below to fit the errors using least squares
\end{itemize}

\begin{equation}
    logy = logA + Bx
\end{equation}

\begin{itemize}
\item
  To find the time constant of error function obtained for the two cases
  using lstsq and compare them
\item
  To plot the two fits obtained and observe them
\end{itemize}

\subsection*{Code:}

\begin{verbatim}
expe = error[500:]
logy = log(expe)
n = linspace(501,Niter,Niter-500)
L1 = ones((Niter-500,2))
L1[:,0]= n

expe1 = error[:]
logy1 = log(expe1)
n1 = linspace(1,Niter,Niter)
R1 = ones((Niter,2))
R1[:,0]= n1

# Fitting the error to an exponential for > 500 iterations
err = linalg.lstsq(L1,logy,rcond=-1)[0] 
error_fit =  err[1] + err[0]*n


# Fitting the error to an exponential
err1= linalg.lstsq(R1,logy1,rcond=-1)[0] 
error_fit1 =  err1[1] + err1[0]*n1


# Plotting the error fits
figure(4)
title('Error in semilog scalefor all iterations')
xlabel(r'$Iterations$')
ylabel(r'$Error$')
plot(linspace(1,Niter,Niter),log(error),"g",label = 'errors')
plot(n,error_fit,"ro",label = 'fit error > 500 iter')
plot(n1,error_fit1,"b--",label = 'fit error')
grid()
legend()
show()

# Plotting the 3D surface plot of potential
fig4 = figure(5) 
ax = p3.Axes3D(fig4)
title('The 3-D Surface Plot of the potential')
ylabel(r'$GROUND$')
surf = ax.plot_surface(X,Y,phi,rstride=1,cstride=1,cmap = cm.jet,label = 'Potential')
show()


\end{verbatim}



\begin{figure}[!tbh]
 \centering
 \includegraphics[scale=0.8]{Figure_154.png}  
 %\includegraphics[scale=0.8]{Figure_55.png}  
 \caption{Semilog plot of Error vs No.of Iterations}
\end{figure}


\newpage

\section*{Task 4 :\\Plotting 3D surface plot and Contour plot of potential }
\begin{itemize}
\item
  To do a 3-D surface plot of the potential.
\item
  To plot contour plot of potential
\item
  And analyse them and to comment about flow of currents
\end{itemize}
\subsection*{Code:}

\begin{verbatim}
 # Plotting the 3D surface plot of potential
fig4 = figure(5) 
ax = p3.Axes3D(fig4)
title('The 3-D Surface Plot of the potential')
ylabel(r'$GROUND$')
surf = ax.plot_surface(X,Y,phi,rstride=1,cstride=1,cmap = cm.jet,label = 'Potential')
show()

# Contour plot of potential
figure(6)
title('Contour plot of the potential') 
xlabel(r'$x$')
ylabel(r'$y$')
plot(x[Z[0]],y[Z[1]],'ro', label = '1V Potential')
clabel(contour(x,y,phi) , inline = True , fontsize = 8)
grid()
legend()
show()
\end{verbatim}

\begin{figure}[!tbh]
 \centering
 \includegraphics[scale=0.8]{Figure_55.png}  
 \caption{3-D Surface potential plot }
\end{figure}
\newpage

\begin{figure}[!tbh]
 \centering
 \includegraphics[scale=0.8]{Figure_56.png}  
 \caption{ Contour plot of potential}
\end{figure}
\newpage


\section*{Task 5 :\\Vector Plot of Currents:}
\begin{itemize}
\item
  To obtain the currents by computing the gradient.
\item
  The actual value of \(\sigma\) does not matter to the shape of the
  current profile, so we set it to unity. Our equations are
\end{itemize}

\begin{equation}
    J_x = -\frac{\partial \phi}{\partial x} 
  \end{equation}

\begin{equation}
    J_y = -\frac{\partial \phi}{\partial y} 
  \end{equation}

\begin{itemize}
\item
  To program this we use these equations as follows:
\end{itemize}

\begin{equation}
        J_{x,ij} = \frac{1}{2}(\phi_{i,j-1} - \phi_{i,j+1}) 
    \end{equation}

\begin{equation}
        J_{y,ij} = \frac{1}{2}(\phi_{i-1,j} - \phi_{i+1,j}) 
    \end{equation}
  
 \subsection*{Code:}
\begin{verbatim}
# Plotting the current vectors using quiver
figure(7) 
title('Quiver plot of the current densities')
xlabel(r'$x$')
ylabel(r'$y$')
plot(x[Z[0]],y[Z[1]],'ro', label = '1V potential')
quiver(y,x,-Jy[::-1,:],-Jx[::-1,:],scale=1.8,scale_units='inches',label="Current density")
grid()
legend()
show()   
\end{verbatim}

  \begin{figure}[!tbh]
   \centering
   \includegraphics[scale=0.8]{Figure_57.png}  
   \caption{Vector plot of current flow}
  \end{figure}



 \section*{Additional :\\\\Heat Map of the conductor:}

As the current flows in the conductor, it heats up. Thus increasing it's
temperature.This \\phenomenon is called Joule Heating.
\\The heat equation is given by :

\begin{equation}
 \kappa \nabla^2{T}=-\frac{1}{\sigma} {|j|^2}   
\end{equation}
 We take,
 \begin{equation}
   \kappa=1,\sigma=1 \ and \ \Delta{x}=1  \ for \ simplicity. 
 \end{equation}

 Thus expanding this equation gives us:

\begin{equation}
 T_{i,j}= \frac{T_{i+1,j}+T_{i-1,j}+T_{i,j+1}+T_{i,j-1}+|J|^2}{4(\Delta{x})^2}  
\end{equation}

\\Thus by updating the temperature \texttt{Niter} times we get a
temperature which converges. \\The boundary condition is that at the
boundary \(\frac{\partial{T}}{\partial{n}}=0\)

\subsection*{Code:}
\begin{verbatim} 
T = zeros((Nx,Ny))
T[:,:] = 300
sigma = 6*(10**7)
kappa = 385
for i in range(Niter):
    T[1:-1,1:-1] = 0.25*(T[1:-1,0:-2] + T[1:-1,2:] + T[0:-2,1:-1] + T[2:,1:-1] + (((Jx**2)[1:-1,1:-1] + (Jy**2)[1:-1,1:-1])*sigma*(16*(10**-8)))/kappa)
    T[1:-1,0]=T[1:-1,1]
    T[1:-1,Nx-1]=T[1:-1,Nx-2]
    T[0,1:-1]=T[1,1:-1]
    T[Z] = 300.0


fig1=figure(4)
ax=p3.Axes3D(fig1)
title('The 3-D surface plot of the temperature')
ax.set_xlabel('x')
ax.set_ylabel('y')
ax.set_zlabel('Temperature')
ax.plot_surface(X, Y, T, rstride=1, cstride=1, cmap=cm.jet,linewidth=0, antialiased=False)
show()
\end{verbatim} 

 \begin{figure}[!tbh]
 \centering
 \includegraphics[scale=0.8]{Figure_58.png}  
 \caption{\textbf{3D surface plot of the Temperature}}
 \end{figure}
   
   
\newpage
  
\subsection*{Code:}
\begin{verbatim}     
J_sq = Jx**2 + Jy**2

figure(9)
title('Contour plot of the heat generated')
cp = contour(-Y,-X,J_sq)
clabel(cp,inline=True,colors='r')
xlabel('x')
ylabel('y')
grid()
show()
\end{verbatim}
 
 \begin{figure}[!tbh]
 \centering
 \includegraphics[scale=0.8]{Figure_159.png}  
 \caption{Counter plot of the Heat Generated}
 \end{figure}
   
\newpage
      
\section*{Conclusion :}\label{results-and-conclusion}
  
  \begin{itemize}
  \item
    To conclude , Most of the current is in the narrow region at the
    bottom.So that is what will get strongly heated.\\
  \item
    Since there is almost no current in the upper region of plate,the
    bottom part of the plate gets hotter and temperature increases in down
    region of the plate.\\
  \item
    And we know that heat generated is from \(\vec{J}.\vec{E}\) (ohmic
    loss) so since \(\vec{J}\) and \(\vec{E}\) are higher in the bottom
    region of the plate, there will more heat generation and temperature
    rise will be present.\\
  \item
    So overall we looked the modelling of the currents in resistor in this
    report ,and we observe that the best method to solve this is to
    increase \(N_x\) and \(N_y\) to very high values(100 or \(\geq\)
    100)and increase the no of iterations too, so that we get accurate
    answers i.e currents in the resistor.\\
  \item
    But the tradeoff is this method of solving is very slow even though we
    use vectorized code because the decrease in errors is very slow w.r.t
    no of iterations.
  \end{itemize}
 
   \end{document}
  
  