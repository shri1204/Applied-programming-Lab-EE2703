\documentclass[11pt, a4paper, twoside]{article}
\usepackage{graphicx}
\usepackage{amsmath}
\usepackage[margin=0.8in]{geometry}
\usepackage{listings}
\usepackage{float}
\usepackage{fancyhdr}
\usepackage{indentfirst}
\usepackage[inline]{enumitem}
\usepackage{xcolor}
\usepackage{minted}
\usemintedstyle{borland}
\usepackage[belowskip=0pt,aboveskip=0pt,font=small,labelfont=small]{caption}
\captionsetup{width=0.9\linewidth}
\setlength\intextsep{0pt}



\title{EE2703: Assignment 6}
\author{KORRA SRIKANTH \\ \small EE19B033}
\date{\today}

\begin{document}	
	
\maketitle % Insert the title, author and date		
  \section*{Introduction}
   

In this assignment, we model a tubelight as a one dimensional space of
gas in which electrons are continually injected at the cathode and
accelerated towards the anode by a constant electric field. The
electrons can ionize material atoms if they achieve a velocity greater
than some threshold, leading to an emission of a photon. This ionization
is modeled as a random process. The tubelight is simulated for a certain
number of timesteps from an initial state of having no electrons. The
results obtained are plotted and studied.


\section*{Defining and Initializing the parameters:}
The tubelight is simulated with the default parameters of \(n=100\),
\(M=5\), \(nk=500\) and \(Msig=0.2\). A threshold speed is \(uo = 7\), and
an ionization probability is \(p=0.5\) are chosen.

\subsection*{Code:}
A function to simulate the tubelight given certain parameters is written
below:
\begin{verbatim}
 if (len(sys.argv)== 7) :
        n = int(sys.argv[1])         # Length of the tube-light.
        M = int(sys.argv[2])         # Number of Electrons injecter per turn.
        nk= int(sys.argv[3])         # Number of turns of simulation.
        uo= int(sys.argv[4])         # Threshold Velocity of Electrons.
        p = float(sys.argv[5])       # Probability of Ionization.
        Msig = float(sys.argv[6])    # Std-deviation of 'M'.
        print("Using user provided Arguments!!")
else :      # Default Values :
        n = 100                      
        M = 5                       
        nk= 500                    
        uo= 7                       
        p = 0.5                     
        Msig = 0.2  
        print("Using Default Arguments!!")

# Initializing the vectors to zeros:
xx = np.zeros(n*M)                   
u  = np.zeros(n*M)
dx = np.zeros(n*M)
# Creating Three empty lists:
X=[]                                
V=[]
I=[]

\end{verbatim}

\section*{Tubelight model and Running the simulation}

If we consider the fact that an electron will collide after a
\(dt\) amount of time, and then accelerate after its collision for the
remaining time period, we need to perform a more accurate update step.
This is done by taking time as the uniformly distributed random
variable. Say \(dt\) is a uniformly distributed random variable between
\(0\) and \(1\). Then, the electron would have traveled an actual
distance of \(dx'\) given by

\[dx_i' = u_i + \frac{1}{2}dt^2\]

as opposed to \(dx_i = u_i + 0.5\)

We update the positions of collisions using this displacement instead.
We also consider that the electrons accelerate after the collision for
the remaining \(1-dt\) period of time. We get the following equations
for position and velocity updates:

\[dx_i'' = \frac{1}{2}(1-dt)^2\]

\[u_{i+1} = 1-dt\]

With the following update rule:

\[xx_{i+1} = xx_i + dx_i' + dx_i''\]

\subsection*{Code:}
\begin{verbatim}
 for k in range(1,nk) :
                 ii = np.where(xx>0)[0]     # indices where electrons are present.                  
                 dx[ii] = u[ii] + 0.5       # updating the displacement.                 
                 xx[ii] = xx[ii] + dx[ii]   # updating the position.                 
                 u[ii]  = u[ii] + 1         # updating the velocity.
                 
                 # indices of electrons which reached the anode:
                 aa = np.where(xx > n)[0]                     
                 xx[aa] = 0        # setting the position to be 0.                           
                 u[aa] = 0         # setting the velocity to be 0.                          
                 dx[aa] = 0        # setting the displacement to be 0.                           

                 # indices of electrons whose velocity is greater than threshold:
                 kk = np.where(u >= uo)[0]                    
                 ll=np.where(np.random.rand(len(kk))<=p)[0] 
                 # kl contains the indices of electrons which undergo collisions:

                 kl=kk[ll] 
                 # Setting the velocity of collided electrons to 0:                 
                 u[kl] = 0                                    

                 # setting position of collided electrons: 
                 xx[kl] = xx[kl] - dx[kl]*np.random.rand()
                 # Updating the Intensity list:
                 I.extend(xx[kl].tolist())   
                 
                 # Injected electrons:   
                 m = int(np.random.randn()*Msig + M)
                 # Empty indices in the light:
                 ee = np.where(xx == 0)[0]                    
                 
                 if len(ee) >= m:                             
                       start = np.random.randint(len(ee)) 
                       xx[ee[start: m+start]] = 1
                       u[ee[start-m:start]] = 0         
                 else :
                       xx[ee] = 1                            
                       u[ee]  = 0
                 filled = np.where(xx > 0)
                 # Updating the Position list:
                 X.extend(xx[filled].tolist()) 
                 # Updating the Velocity list:               
                 V.extend(u[filled].tolist()
\end{verbatim}

\subsubsection*{Plots:}
\textbf{Using these updates, we get the following plots:}
\begin{figure}[!tbh]
      \centering
      \includegraphics[scale=0.8]{Figure_60.png}  
       \caption{Electron Density Plot}
 \end{figure}

\begin{figure}[!tbh]
      \centering
      \includegraphics[scale=0.8]{Figure_61.png}  
      \caption{Emission Intensity Plot }
 \end{figure}

\begin{figure}[!tbh]
      \centering
      \includegraphics[scale=0.8]{Figure_62.png}  
      \caption{Electron Phase space Plot}
 \end{figure}
 


\newpage
\subsection*{Discussions}
\begin{itemize}
\item
  The electron density is peaked at the initial parts of the tubelight
  as the electrons are gaining speed here and are not above the
  threshold. This means that the peaks are the positions of the
  electrons at the first few timesteps they experience.
\item
  The peaks slowly smoothen out as \(x\) increases beyond \(19\). This
  is because the electrons achieve a threshold speed of \(7\) only after
  traversing a distance of \(19\) units. This means that they start
  ionizing the gas atoms and lose their speed due to an inelastic
  collision.
\item
  The emission intensity also shows peaks which get diffused as \(x\)
  increases. This is due the same reason as above. Most of the electrons
  reach the threshold at roughly the same positions, leading to peaks in
  the number of photons emitted there.
\item
  This phenomenon can also be seen in the phase space plot. Firstly, the
  velocities are restricted to discrete values, as the acceleration is
  set to \(1\), and we are not yet performing accurate velocity updates
  after collisions.
\item
  One trajectory is separated from the rest of plot. This corresponds to
  those electrons which travel until the anode without suffering any
  inelastic collisions with gas atoms. This can be seen by noticing that
  the trajectory is parabolic. This means that
  \(v \space = k \sqrt{x}\), which is precisely the case for a particle
  moving with constant acceleration.
\item
  The rest of the plot corresponds to the trajectories of those
  electrons which have suffered at least one collision with an atom.
  Since the collisions can occur over a continuous range of positions,
  the trajectories encompass all possible positions after \(x=19\).
\end{itemize}

   \section*{Conclusion:}\label{conclusion}
\begin{itemize}
\item
  Since the threshold speed is much lower in the second set of
  parameters, photon emission starts occuring from a much lower value of
  x. This means that the electron density is more evenly spread out. It
  also means that the emission intensity is very smooth, and the
  emission peaks are very diffused.
\item
  Since the probability of ionization is very high, total emission
  intensity is also relatively higher compared to the first case.
\item
  We can conclude from the above observations that a gas which has a
  lower threshold velocity and a higher ionization probability is better
  suited for use in a tubelight, as it provides more uniform and a
  higher amount of photon emission intensity.
\item
  Coming to the case where the ionization probability is 1, we observe
  that the emission instensity consists of distinct peaks. The reason
  that these peaks are diffused is that we perform the actual collision
  at some time instant within the interval between two time steps. This
  also explains the slightly diffused phase plot as well.

  \end{itemize}

	\end{document}
